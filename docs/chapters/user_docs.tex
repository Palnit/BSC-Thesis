\chapter{User Documentation}

In this chapter we will explore the usage of the program from the perspective of and end user. We define the minimum requirements for the programs running, provide ways for installation and show example usages of the program.

\section{Program requirements}

The program can run on any 64 bit processor and is prepared to run on Windows and Linux operating system. The program can run with only a \ac{CPU} but for it to work with OpenGl it needs a \ac{GPU} that is capable to run atleast 1024 concurrent threads. The CUDA functionality requires an NVIDIA \ac{GPU} whit the same concurrent thread requirement.

\section{Installation guide}

The program requires no installation for an end user. You only need to download the provided compressed binaries for your operating system from the following website \hyperref{https://github.com/Palnit/BSC-Thesis}{https://github.com/Palnit/BSC-Thesis} and unzip the binary and run. If you want to build it for your self with CUDA capability you need to download the CUDA SDK from \hyperref{test}{test} and CMAKE with a cpp build system e.g.: ninja, make and a compiler e.g.: g++, MSVC, Clangd. 



