\chapter{Introduction}
\label{chap:intro}

The aim of this thesis is to show the differences of edge detectors and the difference of two main \ac{GPGPU} programming library's namely OpenCL and CUDA.

This thesis 	topic is mainly aims to answer the question of what \ac{GPGPU} library is the better and exactly how important it is to do graphics and or computer vision related tasks on the \ac{GPU}. While we know by the structure of how these libraries operate that CUDA will be the most optimal of the two we cannot be sure of this before testing this out.

The reason I choose this topic is because while I was studying \ac{GPGPU} during my university years I discovered that there's almost no evidence proving that CUDA is faster. While there's many discussions online about while CUDA is faster there have been no research or tests done on this topic thus I took it upon my self to test it. I choose edge detection as the test computing they are easy to understand and the algorithms for them are already optimized. I use two edge detection algorithms one is by far one of the most used \ac{Canny}. The other algorithm is the \ac{DoG} which is a far more simple algorithm than \ac{Canny} and it should finish a lot faster.

This thesis is organised into four chapters. The first chapter is this one witch serves as an introduction to the topic and goals of this thesis.

The second chapter is the user documentation which serves a way to introduce the program to an end user. Demonstrating how to install, what are the requirements for running and example usages.

The third chapter is a developer documentation explaining how the code works what tests were conducted, what was our plan going into development, what technologies we used, and generally the inner workings of the program.

And the forth and final chapter will be a summary of our findings during the thesis. We will review what we can expect if we conduct the test on different machines and methods to optimize the code better and what other edge detectors we could implement.